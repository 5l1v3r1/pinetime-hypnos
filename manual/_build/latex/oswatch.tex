%% Generated by Sphinx.
\def\sphinxdocclass{report}
\documentclass[letterpaper,10pt,english]{sphinxmanual}
\ifdefined\pdfpxdimen
   \let\sphinxpxdimen\pdfpxdimen\else\newdimen\sphinxpxdimen
\fi \sphinxpxdimen=.75bp\relax

\PassOptionsToPackage{warn}{textcomp}
\usepackage[utf8]{inputenc}
\ifdefined\DeclareUnicodeCharacter
% support both utf8 and utf8x syntaxes
  \ifdefined\DeclareUnicodeCharacterAsOptional
    \def\sphinxDUC#1{\DeclareUnicodeCharacter{"#1}}
  \else
    \let\sphinxDUC\DeclareUnicodeCharacter
  \fi
  \sphinxDUC{00A0}{\nobreakspace}
  \sphinxDUC{2500}{\sphinxunichar{2500}}
  \sphinxDUC{2502}{\sphinxunichar{2502}}
  \sphinxDUC{2514}{\sphinxunichar{2514}}
  \sphinxDUC{251C}{\sphinxunichar{251C}}
  \sphinxDUC{2572}{\textbackslash}
\fi
\usepackage{cmap}
\usepackage[T1]{fontenc}
\usepackage{amsmath,amssymb,amstext}
\usepackage{babel}



\usepackage{times}
\expandafter\ifx\csname T@LGR\endcsname\relax
\else
% LGR was declared as font encoding
  \substitutefont{LGR}{\rmdefault}{cmr}
  \substitutefont{LGR}{\sfdefault}{cmss}
  \substitutefont{LGR}{\ttdefault}{cmtt}
\fi
\expandafter\ifx\csname T@X2\endcsname\relax
  \expandafter\ifx\csname T@T2A\endcsname\relax
  \else
  % T2A was declared as font encoding
    \substitutefont{T2A}{\rmdefault}{cmr}
    \substitutefont{T2A}{\sfdefault}{cmss}
    \substitutefont{T2A}{\ttdefault}{cmtt}
  \fi
\else
% X2 was declared as font encoding
  \substitutefont{X2}{\rmdefault}{cmr}
  \substitutefont{X2}{\sfdefault}{cmss}
  \substitutefont{X2}{\ttdefault}{cmtt}
\fi


\usepackage[Bjarne]{fncychap}
\usepackage{sphinx}

\fvset{fontsize=\small}
\usepackage{geometry}

% Include hyperref last.
\usepackage{hyperref}
% Fix anchor placement for figures with captions.
\usepackage{hypcap}% it must be loaded after hyperref.
% Set up styles of URL: it should be placed after hyperref.
\urlstyle{same}

\usepackage{sphinxmessages}




\title{open source watch Documentation}
\date{Dec 08, 2019}
\release{1.0.0}
\author{jj}
\newcommand{\sphinxlogo}{\vbox{}}
\renewcommand{\releasename}{Release}
\makeindex
\begin{document}

\pagestyle{empty}
\sphinxmaketitle
\pagestyle{plain}
\sphinxtableofcontents
\pagestyle{normal}
\phantomsection\label{\detokenize{index::doc}}
\noindent\sphinxincludegraphics{{zephyr_logo}.png}

\noindent\sphinxincludegraphics{{PineTime-830x400}.png}



\begin{sphinxVerbatim}[commandchars=\\\{\}]
\PYG{n}{this} \PYG{n}{document} \PYG{n}{describes} \PYG{n}{the} \PYG{n}{installation} \PYG{n}{of} \PYG{n}{zephyr} \PYG{n}{RTOS} \PYG{n}{on} \PYG{n}{the} \PYG{n}{PineTime} \PYG{n}{smartwatch}\PYG{o}{.}

\PYG{n}{https}\PYG{p}{:}\PYG{o}{/}\PYG{o}{/}\PYG{n}{wiki}\PYG{o}{.}\PYG{n}{pine64}\PYG{o}{.}\PYG{n}{org}\PYG{o}{/}\PYG{n}{index}\PYG{o}{.}\PYG{n}{php}\PYG{o}{/}\PYG{n}{PineTime}

\PYG{n}{It} \PYG{n}{should} \PYG{n}{be} \PYG{n}{applicable} \PYG{n}{on} \PYG{n}{other} \PYG{n}{nordic} \PYG{n}{nrf52832} \PYG{n}{based} \PYG{n}{watches} \PYG{p}{(}\PYG{n}{Desay} \PYG{n}{D6}\PYG{o}{.}\PYG{o}{.}\PYG{o}{.}\PYG{o}{.}\PYG{p}{)}\PYG{o}{.}
\end{sphinxVerbatim}

\begin{sphinxVerbatim}[commandchars=\\\{\}]
\PYG{n}{the} \PYG{n}{approach} \PYG{o+ow}{in} \PYG{n}{this} \PYG{n}{manual} \PYG{o+ow}{is} \PYG{n}{to} \PYG{n}{get} \PYG{n}{quick} \PYG{n}{results} \PYG{p}{:}
    \PYG{o}{\PYGZhy{}} \PYG{n}{minimal} \PYG{n}{effort} \PYG{n}{install}
    \PYG{o}{\PYGZhy{}} \PYG{k}{try} \PYG{n}{out} \PYG{n}{the} \PYG{n}{samples}
    \PYG{o}{\PYGZhy{}} \PYG{n}{inspire} \PYG{n}{you} \PYG{n}{to} \PYG{n}{modify} \PYG{o+ow}{and} \PYG{n}{enhance}
\end{sphinxVerbatim}
\begin{description}
\item[{suggestion :}] \leavevmode\begin{itemize}
\item {} 
install zephyr, \sphinxurl{https://docs.zephyrproject.org}

\item {} 
copy the board definition

\item {} 
try some examples

\item {} 
try out bluetooth

\item {} 
try out the display

\end{itemize}

\end{description}


\chapter{About}
\label{\detokenize{about:about}}\label{\detokenize{about::doc}}
I got a pinetime development kit very early.

I would like to thank the folks from \sphinxurl{https://www.pine64.org/}.

I like to hack stuff, and I like the idea behind Open Source.

The smartwatches I hacked, contained microcontrollers from Nordic Semiconductor.

A lot of resources exist for this breed.

It is an Arm based, 32bit microcontroller with a lot of flash and RAM memory.

In fact it is a small computer on your wrist, with a battery and screen, and capable of bluetooth 4+ wireless communication.

\begin{sphinxVerbatim}[commandchars=\\\{\}]
\PYG{n}{A} \PYG{n}{word} \PYG{n}{of} \PYG{n}{warning}\PYG{p}{:} \PYG{n}{this} \PYG{o+ow}{is} \PYG{n}{work} \PYG{o+ow}{in} \PYG{n}{progress}\PYG{o}{.}
\PYG{n}{You}\PYG{l+s+s1}{\PYGZsq{}}\PYG{l+s+s1}{re likely to have a better skillset then me.}
\PYG{n}{You} \PYG{n}{are} \PYG{n}{invited} \PYG{n}{to} \PYG{n}{add} \PYG{n}{the} \PYG{n}{missing} \PYG{n}{pieces} \PYG{o+ow}{and} \PYG{n}{to} \PYG{n}{improve} \PYG{n}{what}\PYG{l+s+s1}{\PYGZsq{}}\PYG{l+s+s1}{s already there.}
\end{sphinxVerbatim}


\section{Todo}
\label{\detokenize{about:todo}}
list with suggestions:
\begin{itemize}
\item {} 
better graphics (lvgl using images and rotating stuff)

\item {} 
NOR flash (here one can store data)

\item {} 
watchdog

\item {} 
DFU (update over bluetooth)

\item {} 
acceleration sensor

\item {} 
heart rate sensor

\item {} 
fun stuff

\item {} 
useless stuff, but somehow cool

\item {} 
applications, e.g. calculator, cycle computer, step counter, heart attack predictor …

\end{itemize}


\section{Fast track}
\label{\detokenize{about:fast-track}}
In this repository you can find modified directories, which you can copy to the zephyrproject directory:
\begin{itemize}
\item {} 
pinetime (board definition -\textgreater{} boards/arm)

\item {} 
st7789v (example -\textgreater{} samples/display)

\item {} 
blinky (example -\textgreater{} samples/basic)

\end{itemize}


\chapter{Install zephyr}
\label{\detokenize{installation:install-zephyr}}\label{\detokenize{installation::doc}}
\sphinxurl{https://docs.zephyrproject.org/latest/getting\_started/index.html}

the documentation describes an installation process under Ubuntu/macOS/Windows

I picked Debian (which is not listed)
…. and soon afterwards ran into trouble

\sphinxtitleref{this behaviour is known as : stuborn or stupid, but I remain convinced it could work}

But even after following the rules, I got a problem with the \sphinxcode{\sphinxupquote{dtc (device tree compiler)}}
\begin{itemize}
\item {} 
I solved this by creating a link from the development-tools to /usr/bin/dtc (here you need to make sure you got a very recent one)

\end{itemize}

\begin{sphinxVerbatim}[commandchars=\\\{\}]
\PYG{g+go}{cd  /root/zephyr\PYGZhy{}sdk\PYGZhy{}0.10.3/sysroots/x86\PYGZus{}64\PYGZhy{}pokysdk\PYGZhy{}linux/usr/bin/}
\PYG{g+go}{mv dtc dtc\PYGZhy{}orig}
\PYG{g+go}{ln \PYGZhy{}s /usr/bin/dtc dtc}
\end{sphinxVerbatim}

\sphinxstylestrong{Note : in order to get the display st7789 Picture-Perfect, you might need a zephyr patch}

have a look at : \sphinxurl{https://github.com/zephyrproject-rtos/zephyr/pull/20570/files}
You will find them in this repo under patches-zephyr.


\chapter{zephyr on the pinetime smartwatch}
\label{\detokenize{blinky:zephyr-on-the-pinetime-smartwatch}}\label{\detokenize{blinky::doc}}

\section{Blinky    example}
\label{\detokenize{blinky:blinky-example}}
\sphinxstylestrong{Note: I think you need to connect the 5V, just connecting the SWD cable (3.3V) is likely not enough to light up the leds}

\begin{sphinxVerbatim}[commandchars=\\\{\}]
\PYG{n}{The} \PYG{n}{watch} \PYG{n}{does} \PYG{o+ow}{not} \PYG{n}{contain} \PYG{n}{a} \PYG{n}{led} \PYG{k}{as} \PYG{n}{such}\PYG{p}{,} \PYG{n}{but} \PYG{n}{it} \PYG{n}{has} \PYG{n}{background} \PYG{n}{leds} \PYG{k}{for} \PYG{n}{the} \PYG{n}{LCD}\PYG{o}{.}

\PYG{n}{Once} \PYG{n}{lit}\PYG{p}{,} \PYG{n}{you} \PYG{n}{can} \PYG{n}{barely} \PYG{n}{see} \PYG{n}{it}\PYG{p}{,} \PYG{n}{cause} \PYG{n}{the} \PYG{n}{screen} \PYG{o+ow}{is} \PYG{n}{black}\PYG{o}{.}
\end{sphinxVerbatim}

\begin{sphinxVerbatim}[commandchars=\\\{\}]
\PYG{g+go}{copy the board definition for the pinetime to the zephyrproject directory}
\PYG{g+gp}{\PYGZdl{}} cp \PYG{o}{(}this repo\PYG{o}{)}pinetime  \PYGZti{}/zephyrproject/zephyr/boards/arm/pinetime

\PYG{g+go}{replace the blinky sample with the one in this repo}
\PYG{g+gp}{\PYGZdl{}} cp \PYG{o}{(}this repo\PYG{o}{)}blinky  \PYGZti{}/zephyrproject/zephyr/samples/basic
\end{sphinxVerbatim}

have a look at the pinetime.dts file, here you see the definition of the background leds.

\begin{sphinxVerbatim}[commandchars=\\\{\}]
\PYG{g+go}{gpios = \PYGZlt{}\PYGZam{}gpio0 14 GPIO\PYGZus{}INT\PYGZus{}ACTIVE\PYGZus{}LOW\PYGZgt{};}
\PYG{g+go}{gpios = \PYGZlt{}\PYGZam{}gpio0 22 GPIO\PYGZus{}INT\PYGZus{}ACTIVE\PYGZus{}LOW\PYGZgt{};}
\PYG{g+go}{gpios = \PYGZlt{}\PYGZam{}gpio0 23 GPIO\PYGZus{}INT\PYGZus{}ACTIVE\PYGZus{}LOW\PYGZgt{};}
\end{sphinxVerbatim}

\sphinxtitleref{building an image, which can be found under the build directory}

\begin{sphinxVerbatim}[commandchars=\\\{\}]
\PYG{g+gp}{\PYGZdl{}} west build \PYGZhy{}p \PYGZhy{}b pinetime samples/basic/blinky
\end{sphinxVerbatim}

once the compilation is completed you can upload the firmware
\textasciitilde{}/zephyrproject/zephyr/build/zephyr/zephyr.bin


\chapter{Reading out the button on the watch}
\label{\detokenize{button:reading-out-the-button-on-the-watch}}\label{\detokenize{button::doc}}
\begin{sphinxVerbatim}[commandchars=\\\{\}]
\PYG{n}{The} \PYG{n}{pinetime} \PYG{n}{does} \PYG{n}{have} \PYG{n}{a} \PYG{n}{button}\PYG{o}{.}
\PYG{n}{I} \PYG{n}{do} \PYG{o+ow}{not} \PYG{n}{have} \PYG{n}{a} \PYG{n}{segger} \PYG{n}{debugging} \PYG{n}{probe}\PYG{o}{.}
\PYG{n}{A} \PYG{n}{way} \PYG{n}{around} \PYG{n}{this}\PYG{p}{,} \PYG{n}{it} \PYG{n}{to} \PYG{n}{put} \PYG{n}{a} \PYG{n}{value} \PYG{o+ow}{in} \PYG{n}{memory} \PYG{n}{at} \PYG{n}{a} \PYG{n}{fixed} \PYG{n}{location}\PYG{o}{.}
\PYG{n}{With} \PYG{n}{openocd} \PYG{n}{you} \PYG{n}{can} \PYG{n}{peek} \PYG{n}{at} \PYG{n}{this} \PYG{n}{memory} \PYG{n}{location}\PYG{o}{.}
\end{sphinxVerbatim}


\section{Building and Running}
\label{\detokenize{button:building-and-running}}
In this repo under samples you will find an adapted basic/button program.
(copy this to the samples/basic directory)

\begin{sphinxVerbatim}[commandchars=\\\{\}]
\PYG{g+go}{west build \PYGZhy{}p \PYGZhy{}b pinetime samples/basic/button}
\end{sphinxVerbatim}

\begin{sphinxadmonition}{note}{Note:}
\#define MY\_REGISTER (*(volatile uint8\_t*)0x2000F000)

in the program you can set values:
MY\_REGISTER=(read button value);

this way you know till whether the code executes
\end{sphinxadmonition}

\begin{sphinxVerbatim}[commandchars=\\\{\}]
\PYG{g+gp}{\PYGZsh{}}telnet \PYG{l+m}{127}.0.0.1 \PYG{l+m}{4444}
\end{sphinxVerbatim}

Peeking

\begin{sphinxVerbatim}[commandchars=\\\{\}]
\PYG{g+go}{once your telnet sessions started:}
\PYG{g+go}{Trying 127.0.0.1...}
\PYG{g+go}{Connected to 127.0.0.1.}
\PYG{g+go}{Escape character is \PYGZsq{}\PYGZca{}]\PYGZsq{}.}
\PYG{g+go}{Open On\PYGZhy{}Chip Debugger}
\PYG{g+gp}{\PYGZgt{}}mdw 0x2000F000 0x1
\PYG{g+go}{0x2000f000: 00000100 (switch pushed)}
\end{sphinxVerbatim}

\sphinxstyleemphasis{Note:}:

\sphinxtitleref{The watch has a button out port (15) and buttin in port (13). You have to set the out-port high. Took me a while to figure this out…}


\chapter{bluetooth (BLE) example}
\label{\detokenize{bluetooth:bluetooth-ble-example}}\label{\detokenize{bluetooth::doc}}

\section{Eddy Stone}
\label{\detokenize{bluetooth:eddy-stone}}
\sphinxstylestrong{Note:}  compile the provided example, so a build directory gets created

\begin{sphinxVerbatim}[commandchars=\\\{\}]
\PYG{g+gp}{\PYGZdl{}} west build \PYGZhy{}p \PYGZhy{}b pinetime samples/bluetooth/eddystone
\end{sphinxVerbatim}

\sphinxcode{\sphinxupquote{this builds an image, which can be found under the build directory}}

I use linux with a bluetoothadapter 4.0.
You need bluez.

\begin{sphinxVerbatim}[commandchars=\\\{\}]
\PYG{g+gp}{\PYGZsh{}}bluetoothctl
\PYG{g+gp}{[bluetooth]\PYGZsh{}}scan on
\end{sphinxVerbatim}

And your Eddy Stone should be visible.

If you have a smartphone, you can download the nrf utilities app from nordic.


\section{Ble Peripheral}
\label{\detokenize{bluetooth:ble-peripheral}}
this example is a demo of the services under bluetooth

first build the image

\begin{sphinxVerbatim}[commandchars=\\\{\}]
\PYG{g+gp}{\PYGZdl{}}  west build \PYGZhy{}p \PYGZhy{}b pinetime samples/bluetooth/peripheral \PYGZhy{}D \PYG{n+nv}{CONF\PYGZus{}FILE}\PYG{o}{=}\PYG{l+s+s2}{\PYGZdq{}prj.conf\PYGZdq{}}
\end{sphinxVerbatim}

the image, can be found under the build directory, and has to be flashed to the pinetime

with linux you can have a look using bluetoothctl

\begin{sphinxVerbatim}[commandchars=\\\{\}]
\PYG{g+gp}{\PYGZsh{}}bluetoothctl
\PYG{g+gp}{[bluetooth]\PYGZsh{}}scan on


\PYG{g+go}{[NEW] Device 60:7C:9E:92:50:C1 Zephyr Peripheral Sample Long}
\PYG{g+go}{once you see your device}
\PYG{g+gp}{[blueooth]\PYGZsh{}}connect \PYG{l+m}{60}:7C:9E:92:50:C1 \PYG{o}{(}the device mac address as displayed\PYG{o}{)}

\PYG{g+go}{then you can already see the services}
\end{sphinxVerbatim}

same thing with the app from nordic, you could try to connect and display value of e.g. heart rate


\section{using Python to read out bluetoothservices}
\label{\detokenize{bluetooth:using-python-to-read-out-bluetoothservices}}
In this repo you will find a python script : readbat.py
In order to use it you need bluez on linux and the python \sphinxtitleref{bluepy} module.

It can be used in conjunction with the peripheral bluetooth demo.
It just reads out the battery level, and prints it.

\begin{sphinxVerbatim}[commandchars=\\\{\}]
\PYG{k+kn}{import} \PYG{n+nn}{binascii}
\PYG{k+kn}{from} \PYG{n+nn}{bluepy}\PYG{n+nn}{.}\PYG{n+nn}{btle} \PYG{k}{import} \PYG{n}{UUID}\PYG{p}{,} \PYG{n}{Peripheral}

\PYG{n}{temp\PYGZus{}uuid} \PYG{o}{=} \PYG{n}{UUID}\PYG{p}{(}\PYG{l+m+mh}{0x2A19}\PYG{p}{)}

\PYG{n}{p} \PYG{o}{=} \PYG{n}{Peripheral}\PYG{p}{(}\PYG{l+s+s2}{\PYGZdq{}}\PYG{l+s+s2}{60:7C:9E:92:50:C1}\PYG{l+s+s2}{\PYGZdq{}}\PYG{p}{,} \PYG{l+s+s2}{\PYGZdq{}}\PYG{l+s+s2}{random}\PYG{l+s+s2}{\PYGZdq{}}\PYG{p}{)}

\PYG{k}{try}\PYG{p}{:}
   \PYG{n}{ch} \PYG{o}{=} \PYG{n}{p}\PYG{o}{.}\PYG{n}{getCharacteristics}\PYG{p}{(}\PYG{n}{uuid}\PYG{o}{=}\PYG{n}{temp\PYGZus{}uuid}\PYG{p}{)}\PYG{p}{[}\PYG{l+m+mi}{0}\PYG{p}{]}
   \PYG{n+nb}{print} \PYG{n}{binascii}\PYG{o}{.}\PYG{n}{b2a\PYGZus{}hex}\PYG{p}{(}\PYG{n}{ch}\PYG{o}{.}\PYG{n}{read}\PYG{p}{(}\PYG{p}{)}\PYG{p}{)}
\PYG{k}{finally}\PYG{p}{:}
    \PYG{n}{p}\PYG{o}{.}\PYG{n}{disconnect}\PYG{p}{(}\PYG{p}{)}
\end{sphinxVerbatim}


\chapter{st7789 display}
\label{\detokenize{display:st7789-display}}\label{\detokenize{display::doc}}

\section{Display    example}
\label{\detokenize{display:display-example}}
\sphinxstylestrong{Note: I think you need to connect the 5V, just connecting the SWD cable (3.3V) is likely not enough to light up the leds}
\sphinxstylestrong{While connecting 5V, do not connect 3.3V}

\begin{sphinxVerbatim}[commandchars=\\\{\}]
\PYG{n}{The} \PYG{n}{watch} \PYG{n}{has} \PYG{n}{background} \PYG{n}{leds} \PYG{k}{for} \PYG{n}{the} \PYG{n}{LCD}\PYG{o}{.}

\PYG{n}{They} \PYG{n}{need} \PYG{n}{to} \PYG{n}{be} \PYG{n}{on} \PYG{p}{(}\PYG{n}{LOW}\PYG{p}{)} \PYG{n}{to} \PYG{n}{visualize} \PYG{n}{the} \PYG{n}{display}\PYG{o}{.}
\end{sphinxVerbatim}

\begin{sphinxVerbatim}[commandchars=\\\{\}]
\PYG{g+go}{replace the display sample with the one in this repo}
\PYG{g+gp}{\PYGZdl{}} cp \PYG{o}{(}this repo\PYG{o}{)}st7789  \PYGZti{}/zephyrproject/zephyr/samples/display
\end{sphinxVerbatim}

\sphinxtitleref{building an image, which can be found under the build directory}

\begin{sphinxVerbatim}[commandchars=\\\{\}]
\PYG{g+gp}{\PYGZdl{}}  west build \PYGZhy{}p \PYGZhy{}b pinetime samples/display/st7789v
\end{sphinxVerbatim}

once the compilation is completed you can upload the firmware
\textasciitilde{}/zephyrproject/zephyr/build/zephyr/zephyr.bin

if all goes well, you should see some coloured squares on your screen


\chapter{LittlevGL Basic Sample}
\label{\detokenize{lvgl:littlevgl-basic-sample}}\label{\detokenize{lvgl:lvgl-sample}}\label{\detokenize{lvgl::doc}}

\section{Overview}
\label{\detokenize{lvgl:overview}}
This sample application displays “Hello World” in the center of the screen
and a counter at the bottom which increments every second.

LittlevGL is a free and open-source graphics library providing everything you need to create embedded GUI with easy-to-use graphical elements, beautiful visual effects and low memory footprint.


\section{Requirements}
\label{\detokenize{lvgl:requirements}}\begin{description}
\item[{definitions can be found under the boards sub-directory}] \leavevmode\begin{itemize}
\item {} 
pinetime.conf

\item {} 
pinetime.overlay

\end{itemize}

\end{description}

The program has been modified to light up the background leds.
Might be unnecessary…
can be found in this repo

\begin{sphinxVerbatim}[commandchars=\\\{\}]
Matching labels are necessary!
pinetime.conf:CONFIG\PYGZus{}LVGL\PYGZus{}DISPLAY\PYGZus{}DEV\PYGZus{}NAME=\PYGZdq{}DISPLAY\PYGZdq{}
pinetime.overlay:               label = \PYGZdq{}DISPLAY\PYGZdq{}; (spi definition)
\end{sphinxVerbatim}


\section{Building and Running}
\label{\detokenize{lvgl:building-and-running}}
Make sure you copied the board definitions.

\begin{sphinxVerbatim}[commandchars=\\\{\}]
\PYG{g+go}{west build \PYGZhy{}p \PYGZhy{}b pinetime samples/gui/lvgl}
\end{sphinxVerbatim}

modifying the font size :

west build -t menuconfig
goto additional libraries / lvgl gui library
(look for fonts, and adapt according to your need)

west build


\section{Todo}
\label{\detokenize{lvgl:todo}}\begin{itemize}
\item {} 
Create a button

\item {} 
touchscreen activation (problem cause zephyr does not support this yet)

\item {} 
lvgl supports lv\_canvas\_rotate(canvas, \&imd\_dsc, angle, x, y, pivot\_x, pivot\_y) should be cool for a clock, chrono…

\end{itemize}


\section{References}
\label{\detokenize{lvgl:references}}
\sphinxurl{https://docs.littlevgl.com/en/html/index.html}

LittlevGL Web Page: \sphinxurl{https://littlevgl.com/}


\chapter{LittlevGL Clock Sample}
\label{\detokenize{clock:littlevgl-clock-sample}}\label{\detokenize{clock:lvgl-clock}}\label{\detokenize{clock::doc}}

\section{Overview}
\label{\detokenize{clock:overview}}
This sample application displays a “clockbackground” in the center of the screen.

LittlevGL is a free and open-source graphics library providing everything you need to create embedded GUI with easy-to-use graphical elements, beautiful visual effects and low memory footprint.

\noindent\sphinxincludegraphics{{clockback}.png}


\section{Requirements}
\label{\detokenize{clock:requirements}}
Make sure the prj.conf contains the following :

\begin{sphinxVerbatim}[commandchars=\\\{\}]
\PYG{n}{CONFIG\PYGZus{}LVGL}\PYG{o}{=}\PYG{n}{y}
\PYG{n}{CONFIG\PYGZus{}LVGL\PYGZus{}OBJ\PYGZus{}IMAGE}\PYG{o}{=}\PYG{n}{y}
\end{sphinxVerbatim}

LitlevGL uses a “c” file to store the image.
You need to convert a jpg, or png image to this c file.
There is an online tool : \sphinxurl{https://littlevgl.com/image-to-c-array}


\section{Building and Running}
\label{\detokenize{clock:building-and-running}}
copy the samples/gui/clock from this repository to the zephyr one.

\begin{sphinxVerbatim}[commandchars=\\\{\}]
\PYG{g+go}{west build \PYGZhy{}p \PYGZhy{}b pinetime samples/gui/clock}
\end{sphinxVerbatim}


\section{Todo}
\label{\detokenize{clock:todo}}\begin{itemize}
\item {} 
create an internal clock (and adjustment mechanism, eg. bluetooth cts)

\item {} 
lvgl supports lv\_canvas\_rotate(canvas, \&imd\_dsc, angle, x, y, pivot\_x, pivot\_y) should be cool for a clock, chrono…

\end{itemize}


\section{References}
\label{\detokenize{clock:references}}
\sphinxurl{https://docs.littlevgl.com/en/html/index.html}

LittlevGL Web Page: \sphinxurl{https://littlevgl.com/}


\chapter{Real Time Clock}
\label{\detokenize{RTC:real-time-clock}}\label{\detokenize{RTC:rtc}}\label{\detokenize{RTC::doc}}

\section{Overview}
\label{\detokenize{RTC:overview}}
This sample application “clock” uses the RTC0 timer.
It uses the counter driver.


\section{Requirements}
\label{\detokenize{RTC:requirements}}
Make sure the prj.conf contains the following :

\begin{sphinxVerbatim}[commandchars=\\\{\}]
\PYG{n}{CONFIG\PYGZus{}COUNTER}\PYG{o}{=}\PYG{n}{y}
\end{sphinxVerbatim}

You need the Kconfig file, which contains :

\begin{sphinxVerbatim}[commandchars=\\\{\}]
\PYG{n}{config} \PYG{n}{COUNTER\PYGZus{}RTC0}
        \PYG{n+nb}{bool}
        \PYG{n}{default} \PYG{n}{y} \PYG{k}{if} \PYG{n}{SOC\PYGZus{}FAMILY\PYGZus{}NRF}
\end{sphinxVerbatim}


\section{Building and Running}
\label{\detokenize{RTC:building-and-running}}
copy the samples/gui/clock from this repository to the zephyr one.

\begin{sphinxVerbatim}[commandchars=\\\{\}]
\PYG{g+go}{west build \PYGZhy{}p \PYGZhy{}b pinetime samples/gui/clock}
\end{sphinxVerbatim}


\section{Todo}
\label{\detokenize{RTC:todo}}\begin{itemize}
\item {} 
time of day clock

\item {} 
setting the time

\end{itemize}


\section{References}
\label{\detokenize{RTC:references}}

\chapter{Serial Nor Flash}
\label{\detokenize{spinor:serial-nor-flash}}\label{\detokenize{spinor::doc}}
\begin{sphinxVerbatim}[commandchars=\\\{\}]
\PYG{g+go}{west build \PYGZhy{}p \PYGZhy{}b pinetime samples/drivers/spi\PYGZus{}flash \PYGZhy{}DCONF=prj.conf}
\end{sphinxVerbatim}


\section{Overview}
\label{\detokenize{spinor:overview}}
This sample application should unlock the serial nor flash memory.
This can be very usefull to store e.g. background for the watch.

compilation problematic ….

/root/zephyrproject/zephyr/samples/drivers/spi\_flash/src/main.c:17:22: error: ‘DT\_INST\_0\_JEDEC\_SPI\_NOR\_LABEL’ undeclared (first use in this function); did you mean ‘DT\_INST\_0\_NORDIC\_NRF\_RTC\_LABEL’?

Turns out this is some problem with the board definition file.

I found it to be very useful to consult the generated dts file.
Here you can check if everything is present.

Guess the dts-file has to be well intended.(structured)

\begin{sphinxVerbatim}[commandchars=\\\{\}]
\PYG{g+go}{vi /root/zephyrproject/zephyr/build/zephyr/include/generated/generated\PYGZus{}dts\PYGZus{}board.conf}
\end{sphinxVerbatim}


\section{Requirements}
\label{\detokenize{spinor:requirements}}
complement the pinetime.dts file with the following (under spi)
\#define JEDEC\_ID\_MACRONIX\_MX25L64      0xC22017

\begin{sphinxVerbatim}[commandchars=\\\{\}]
\PYG{g+go}{\PYGZam{}spi0 \PYGZob{}}
\PYG{g+go}{  compatible = \PYGZdq{}nordic,nrf\PYGZhy{}spi\PYGZdq{};}
\PYG{g+go}{  status = \PYGZdq{}okay\PYGZdq{};}
\PYG{g+go}{  sck\PYGZhy{}pin = \PYGZlt{}2\PYGZgt{};}
\PYG{g+go}{  mosi\PYGZhy{}pin = \PYGZlt{}3\PYGZgt{};}
\PYG{g+go}{  miso\PYGZhy{}pin = \PYGZlt{}4\PYGZgt{};}
\PYG{g+go}{  cs\PYGZhy{}gpios = \PYGZlt{}\PYGZam{}gpio0 27 0\PYGZgt{},\PYGZlt{}\PYGZam{}gpio0 5 0\PYGZgt{};}
\PYG{g+go}{  st7789v@0 \PYGZob{}}
\PYG{g+go}{          compatible = \PYGZdq{}sitronix,st7789v\PYGZdq{};}
\PYG{g+go}{          label = \PYGZdq{}DISPLAY\PYGZdq{};}
\PYG{g+go}{          spi\PYGZhy{}max\PYGZhy{}frequency = \PYGZlt{}8000000\PYGZgt{};}
\PYG{g+go}{          reg = \PYGZlt{}0\PYGZgt{};}
\PYG{g+go}{          cmd\PYGZhy{}data\PYGZhy{}gpios = \PYGZlt{}\PYGZam{}gpio0 18 0\PYGZgt{};}
\PYG{g+go}{          reset\PYGZhy{}gpios = \PYGZlt{}\PYGZam{}gpio0 26 0\PYGZgt{};}
\PYG{g+go}{          width = \PYGZlt{}240\PYGZgt{};}
\PYG{g+go}{          height = \PYGZlt{}240\PYGZgt{};}
\PYG{g+go}{          x\PYGZhy{}offset = \PYGZlt{}0\PYGZgt{};}
\PYG{g+go}{          y\PYGZhy{}offset = \PYGZlt{}0\PYGZgt{};}
\PYG{g+go}{          vcom = \PYGZlt{}0x19\PYGZgt{};}
\PYG{g+go}{          gctrl = \PYGZlt{}0x35\PYGZgt{};}
\PYG{g+go}{          vrhs = \PYGZlt{}0x12\PYGZgt{};}
\PYG{g+go}{          vdvs = \PYGZlt{}0x20\PYGZgt{};}
\PYG{g+go}{          mdac = \PYGZlt{}0x00\PYGZgt{};}
\PYG{g+go}{          gamma = \PYGZlt{}0x01\PYGZgt{};}
\PYG{g+go}{          colmod = \PYGZlt{}0x05\PYGZgt{};}
\PYG{g+go}{          lcm = \PYGZlt{}0x2c\PYGZgt{};}
\PYG{g+go}{          porch\PYGZhy{}param = [0c 0c 00 33 33];}
\PYG{g+go}{          cmd2en\PYGZhy{}param = [5a 69 02 01];}
\PYG{g+go}{          pwctrl1\PYGZhy{}param = [a4 a1];}
\PYG{g+go}{          pvgam\PYGZhy{}param = [D0 04 0D 11 13 2B 3F 54 4C 18 0D 0B 1F 23];}
\PYG{g+go}{          nvgam\PYGZhy{}param = [D0 04 0C 11 13 2C 3F 44 51 2F 1F 1F 20 23];}
\PYG{g+go}{          ram\PYGZhy{}param = [00 F0];}
\PYG{g+go}{          rgb\PYGZhy{}param = [CD 08 14];}

\PYG{g+go}{  \PYGZcb{};}

\PYG{g+go}{  mx25r64: mx25r6435f@1 \PYGZob{}}
\PYG{g+go}{          compatible = \PYGZdq{}jedec,spi\PYGZhy{}nor\PYGZdq{};}
\PYG{g+go}{          reg = \PYGZlt{}1\PYGZgt{};}
\PYG{g+go}{          spi\PYGZhy{}max\PYGZhy{}frequency = \PYGZlt{}1000000\PYGZgt{};}
\PYG{g+go}{          label = \PYGZdq{}MX25R64\PYGZdq{};}
\PYG{g+go}{          jedec\PYGZhy{}id = [0b 40 16];}
\PYG{g+go}{          size = \PYGZlt{}67108864\PYGZgt{};}
\PYG{g+go}{          has\PYGZhy{}be32k;}
\PYG{g+go}{  \PYGZcb{};}
\end{sphinxVerbatim}


\section{Building and Running}
\label{\detokenize{spinor:building-and-running}}
\begin{sphinxVerbatim}[commandchars=\\\{\}]
\PYG{g+go}{west build \PYGZhy{}p \PYGZhy{}b pinetime samples/drivers/spi\PYGZus{}flash}
\end{sphinxVerbatim}


\section{Todo}
\label{\detokenize{spinor:todo}}\begin{quote}
\begin{itemize}
\item {} 
detect ID memory  : it is not the macronix one as suggestion on the pinetime website

\end{itemize}

I found the following : jedec-id = {[}0b 40 16{]}; (OK: can execute sample program)
\begin{itemize}
\item {} 
create working board definition (OK: see above)

\end{itemize}
\end{quote}


\section{References}
\label{\detokenize{spinor:references}}
\sphinxurl{http://files.pine64.org/doc/datasheet/pinetime/MX25L6433F,\%203V,\%2064Mb,\%20v1.6.pdf}


\chapter{sensors on the I2C bus}
\label{\detokenize{sensors:sensors-on-the-i2c-bus}}\label{\detokenize{sensors::doc}}
0x18: Accelerometer: BMA423-DS000
\sphinxurl{https://github.com/BoschSensortec/BMA423-Sensor-API}

0x44: Heart Rate Sensor: HRS3300\_Heart

0x15: Touch Controller: Hynitron CST816S Touch Controller


\chapter{configuring I2C}
\label{\detokenize{i2c:configuring-i2c}}\label{\detokenize{i2c::doc}}

\section{board level definitions}
\label{\detokenize{i2c:board-level-definitions}}
\begin{sphinxVerbatim}[commandchars=\\\{\}]
\PYG{n}{under} \PYG{n}{boards}\PYG{o}{/}\PYG{n}{arm}\PYG{o}{/}\PYG{n}{pinetime} \PYG{n}{are} \PYG{n}{the} \PYG{n}{board} \PYG{n}{definitions}
\PYG{o}{\PYGZhy{}} \PYG{n}{pinetime}\PYG{o}{.}\PYG{n}{dts}
\PYG{o}{\PYGZhy{}} \PYG{n}{pinetime\PYGZus{}defconfig}


\PYG{n}{The} \PYG{n}{sensors} \PYG{o+ow}{in} \PYG{n}{the} \PYG{n}{pintime} \PYG{n}{use} \PYG{n}{the} \PYG{n}{I2C} \PYG{n}{bus}\PYG{o}{.}

\PYG{o}{\PYGZam{}}\PYG{n}{i2c1} \PYG{p}{\PYGZob{}}
        \PYG{n}{compatible} \PYG{o}{=} \PYG{l+s+s2}{\PYGZdq{}}\PYG{l+s+s2}{nordic,nrf\PYGZhy{}twi}\PYG{l+s+s2}{\PYGZdq{}}\PYG{p}{;}
        \PYG{n}{status} \PYG{o}{=} \PYG{l+s+s2}{\PYGZdq{}}\PYG{l+s+s2}{okay}\PYG{l+s+s2}{\PYGZdq{}}\PYG{p}{;}
        \PYG{n}{sda}\PYG{o}{\PYGZhy{}}\PYG{n}{pin} \PYG{o}{=} \PYG{o}{\PYGZlt{}}\PYG{l+m+mi}{6}\PYG{o}{\PYGZgt{}}\PYG{p}{;}
        \PYG{n}{scl}\PYG{o}{\PYGZhy{}}\PYG{n}{pin} \PYG{o}{=} \PYG{o}{\PYGZlt{}}\PYG{l+m+mi}{7}\PYG{o}{\PYGZgt{}}\PYG{p}{;}

      \PYG{p}{\PYGZcb{}}\PYG{p}{;}
\end{sphinxVerbatim}


\section{development trajectory}
\label{\detokenize{i2c:development-trajectory}}
The final goal is to use the accel-sensor in the watch (BMA423), which does not exist yet.
In order to minimize the effort:
\begin{itemize}
\item {} 
we’ll use something that looks like it (ADXL372), because there exists an example.

\item {} 
next we adapt it to use the existing BMA280 sensor (under drivers/sensor)

\item {} 
finally we create a driver for the BMA423, based upon the BMA280

\end{itemize}


\section{defining an I2C sensor}
\label{\detokenize{i2c:defining-an-i2c-sensor}}
\begin{sphinxVerbatim}[commandchars=\\\{\}]
\PYG{n}{under} \PYG{n}{samples}\PYG{o}{/}\PYG{n}{sensor}\PYG{o}{/}\PYG{n}{axl372} \PYG{n}{we} \PYG{n}{create} \PYG{p}{:} \PYG{l+s+s2}{\PYGZdq{}}\PYG{l+s+s2}{pinetime.overlay}\PYG{l+s+s2}{\PYGZdq{}}
\PYG{o}{\PYGZam{}}\PYG{n}{i2c1} \PYG{p}{\PYGZob{}}
    \PYG{n}{status} \PYG{o}{=} \PYG{l+s+s2}{\PYGZdq{}}\PYG{l+s+s2}{okay}\PYG{l+s+s2}{\PYGZdq{}}\PYG{p}{;}
    \PYG{n}{clock}\PYG{o}{\PYGZhy{}}\PYG{n}{frequency} \PYG{o}{=} \PYG{o}{\PYGZlt{}}\PYG{n}{I2C\PYGZus{}BITRATE\PYGZus{}STANDARD}\PYG{o}{\PYGZgt{}}\PYG{p}{;}
    \PYG{n}{adxl372}\PYG{n+nd}{@18} \PYG{p}{\PYGZob{}}
               \PYG{n}{compatible} \PYG{o}{=} \PYG{l+s+s2}{\PYGZdq{}}\PYG{l+s+s2}{adi,adxl372}\PYG{l+s+s2}{\PYGZdq{}}\PYG{p}{;}
               \PYG{n}{reg} \PYG{o}{=} \PYG{o}{\PYGZlt{}}\PYG{l+m+mh}{0x18}\PYG{o}{\PYGZgt{}}\PYG{p}{;}
               \PYG{n}{label} \PYG{o}{=} \PYG{l+s+s2}{\PYGZdq{}}\PYG{l+s+s2}{ADXL372}\PYG{l+s+s2}{\PYGZdq{}}\PYG{p}{;}
               \PYG{n}{int1}\PYG{o}{\PYGZhy{}}\PYG{n}{gpios} \PYG{o}{=} \PYG{o}{\PYGZlt{}}\PYG{o}{\PYGZam{}}\PYG{n}{gpio0} \PYG{l+m+mi}{8} \PYG{l+m+mi}{0}\PYG{o}{\PYGZgt{}}\PYG{p}{;}
               \PYG{p}{\PYGZcb{}}\PYG{p}{;}
    \PYG{p}{\PYGZcb{}}\PYG{p}{;}
\end{sphinxVerbatim}

\sphinxstylestrong{note: this gets somehow merged with the board definition pinetime.dts}

\begin{sphinxVerbatim}[commandchars=\\\{\}]
\PYG{n}{In} \PYG{n}{the} \PYG{l+s+s2}{\PYGZdq{}}\PYG{l+s+s2}{prj.conf}\PYG{l+s+s2}{\PYGZdq{}} \PYG{n}{file} \PYG{n}{we} \PYG{n}{define} \PYG{n}{the} \PYG{n}{sensor}

\PYG{n}{CONFIG\PYGZus{}STDOUT\PYGZus{}CONSOLE}\PYG{o}{=}\PYG{n}{y}
\PYG{n}{CONFIG\PYGZus{}LOG}\PYG{o}{=}\PYG{n}{y}
\PYG{n}{CONFIG\PYGZus{}I2C}\PYG{o}{=}\PYG{n}{y}
\PYG{n}{CONFIG\PYGZus{}SENSOR}\PYG{o}{=}\PYG{n}{y}
\PYG{n}{CONFIG\PYGZus{}ADXL372}\PYG{o}{=}\PYG{n}{y}
\PYG{n}{CONFIG\PYGZus{}ADXL372\PYGZus{}I2C}\PYG{o}{=}\PYG{n}{y}
\PYG{n}{CONFIG\PYGZus{}SENSOR\PYGZus{}LOG\PYGZus{}LEVEL\PYGZus{}WRN}\PYG{o}{=}\PYG{n}{y}
\end{sphinxVerbatim}

\sphinxstylestrong{note: this gets somehow merged with the board definition pinetime\_defconfig}


\section{compiling the sample}
\label{\detokenize{i2c:compiling-the-sample}}
\begin{sphinxVerbatim}[commandchars=\\\{\}]
\PYG{n}{west} \PYG{n}{build} \PYG{o}{\PYGZhy{}}\PYG{n}{p} \PYG{o}{\PYGZhy{}}\PYG{n}{b} \PYG{n}{pinetime} \PYG{n}{samples}\PYG{o}{/}\PYG{n}{sensor}\PYG{o}{/}\PYG{n}{adxl372} \PYG{o}{\PYGZhy{}}\PYG{n}{DCONF}\PYG{o}{=}\PYG{n}{prj}\PYG{o}{.}\PYG{n}{conf}
\end{sphinxVerbatim}


\chapter{Bosch BMA280}
\label{\detokenize{bma280:bosch-bma280}}\label{\detokenize{bma280::doc}}
\begin{sphinxVerbatim}[commandchars=\\\{\}]
\PYG{g+go}{west build \PYGZhy{}p \PYGZhy{}b pinetime samples/drivers/bma280}
\end{sphinxVerbatim}


\section{Overview}
\label{\detokenize{bma280:overview}}
This sample application mimics the presence of a bosch, bma280 accel sensor.
For this sensor exists a driver in zephyr, but no sample.
Remember, I’m not a zephyr expert and am learning on the way.


\section{Requirements}
\label{\detokenize{bma280:requirements}}
complement the pinetime.dts file with the following (under samples/sensor/bma280)

\begin{sphinxVerbatim}[commandchars=\\\{\}]
\PYG{g+go}{\PYGZam{}i2c1 \PYGZob{}}
\PYG{g+go}{        bma280@18 \PYGZob{}}
\PYG{g+go}{                   compatible = \PYGZdq{}bosch,bma280\PYGZdq{};}
\PYG{g+go}{                   reg = \PYGZlt{}0x18\PYGZgt{};}
\PYG{g+go}{                   label = \PYGZdq{}BMA280\PYGZdq{};}
\PYG{g+go}{                   int1\PYGZhy{}gpios = \PYGZlt{}\PYGZam{}gpio0 8 0\PYGZgt{};}
\PYG{g+go}{                  \PYGZcb{};}
\PYG{g+go}{       \PYGZcb{};}
\end{sphinxVerbatim}

Create a file: \sphinxtitleref{/dts/bindings/sensor/bosch,bma280-i2c.yaml}.
Which contains:

\begin{sphinxVerbatim}[commandchars=\\\{\}]
\PYG{n}{compatible}\PYG{p}{:} \PYG{l+s+s2}{\PYGZdq{}}\PYG{l+s+s2}{bosch,bma280}\PYG{l+s+s2}{\PYGZdq{}}
\PYG{n}{include}\PYG{p}{:} \PYG{n}{i2c}\PYG{o}{\PYGZhy{}}\PYG{n}{device}\PYG{o}{.}\PYG{n}{yaml}
\PYG{n}{properties}\PYG{p}{:}
       \PYG{n}{int1}\PYG{o}{\PYGZhy{}}\PYG{n}{gpios}\PYG{p}{:}
       \PYG{n+nb}{type}\PYG{p}{:} \PYG{n}{phandle}\PYG{o}{\PYGZhy{}}\PYG{n}{array}
       \PYG{n}{required}\PYG{p}{:} \PYG{n}{false}
\end{sphinxVerbatim}


\section{Building and Running}
\label{\detokenize{bma280:building-and-running}}

\section{Todo}
\label{\detokenize{bma280:todo}}\begin{itemize}
\item {} 
since no serial port and no J-LINK, I have to print messages to the screen or trough bluetooth serial (which does not exist, or I haven’t found it yet ;) )

\end{itemize}


\section{References}
\label{\detokenize{bma280:references}}

\chapter{Touchscreen Hynitron}
\label{\detokenize{touchscreen:touchscreen-hynitron}}\label{\detokenize{touchscreen::doc}}
\begin{sphinxVerbatim}[commandchars=\\\{\}]
\PYG{g+go}{git clone https://github.com/lupyuen/hynitron\PYGZus{}i2c\PYGZus{}cst0xxse}
\end{sphinxVerbatim}


\section{Overview}
\label{\detokenize{touchscreen:overview}}
this does not exist yet in zephyr, but there is work in progress
\sphinxurl{https://github.com/zephyrproject-rtos/zephyr/pull/16119}


\section{Requirements}
\label{\detokenize{touchscreen:requirements}}
complement the pinetime.dts file with the following (under samples/sensor/bma280)

\begin{sphinxVerbatim}[commandchars=\\\{\}]
\PYG{g+go}{\PYGZam{}i2c1 \PYGZob{}}
\PYG{g+go}{        touch@18 \PYGZob{}}
\PYG{g+go}{                  \PYGZcb{};}
\PYG{g+go}{       \PYGZcb{};}
\end{sphinxVerbatim}

Create a file: \sphinxtitleref{/dts/bindings/sensor/touch.yaml}.
Which contains:

\begin{sphinxVerbatim}[commandchars=\\\{\}]
\PYG{n}{compatible}\PYG{p}{:} \PYG{l+s+s2}{\PYGZdq{}}\PYG{l+s+s2}{touch}\PYG{l+s+s2}{\PYGZdq{}}
\PYG{n}{include}\PYG{p}{:} \PYG{n}{i2c}\PYG{o}{\PYGZhy{}}\PYG{n}{device}\PYG{o}{.}\PYG{n}{yaml}
\PYG{n}{properties}\PYG{p}{:}
       \PYG{n}{int1}\PYG{o}{\PYGZhy{}}\PYG{n}{gpios}\PYG{p}{:}
       \PYG{n+nb}{type}\PYG{p}{:} \PYG{n}{phandle}\PYG{o}{\PYGZhy{}}\PYG{n}{array}
       \PYG{n}{required}\PYG{p}{:} \PYG{n}{false}
\end{sphinxVerbatim}


\section{Building and Running}
\label{\detokenize{touchscreen:building-and-running}}

\section{Todo}
\label{\detokenize{touchscreen:todo}}\begin{quote}

-create touchscreen driver
-create sample
\end{quote}


\section{References}
\label{\detokenize{touchscreen:references}}

\chapter{Menuconfig}
\label{\detokenize{menuconfig:menuconfig}}\label{\detokenize{menuconfig::doc}}

\section{Zephyr is like linux}
\label{\detokenize{menuconfig:zephyr-is-like-linux}}
\sphinxstylestrong{Note:}  to get a feel, compile a program, for example

\begin{sphinxVerbatim}[commandchars=\\\{\}]
\PYG{g+go}{west build \PYGZhy{}p \PYGZhy{}b pinetime samples/bluetooth/peripheral \PYGZhy{}D CONF\PYGZus{}FILE=\PYGZdq{}prj.conf\PYGZdq{}}
\end{sphinxVerbatim}

\sphinxcode{\sphinxupquote{the pinetime contains an external 32Kz crystal}}
now you can have a look in the configurationfile (and modify if needed)

\begin{sphinxVerbatim}[commandchars=\\\{\}]
\PYG{g+gp}{\PYGZdl{}} west build \PYGZhy{}t menuconfig
\end{sphinxVerbatim}

\begin{sphinxVerbatim}[commandchars=\\\{\}]
    \PYG{n}{Modules}  \PYG{o}{\PYGZhy{}}\PYG{o}{\PYGZhy{}}\PYG{o}{\PYGZhy{}}\PYG{o}{\PYGZgt{}}
    \PYG{n}{Board} \PYG{n}{Selection} \PYG{p}{(}\PYG{n}{nRF52832}\PYG{o}{\PYGZhy{}}\PYG{n}{MDK}\PYG{p}{)}  \PYG{o}{\PYGZhy{}}\PYG{o}{\PYGZhy{}}\PYG{o}{\PYGZhy{}}\PYG{o}{\PYGZgt{}}
    \PYG{n}{Board} \PYG{n}{Options}  \PYG{o}{\PYGZhy{}}\PYG{o}{\PYGZhy{}}\PYG{o}{\PYGZhy{}}\PYG{o}{\PYGZgt{}}
    \PYG{n}{SoC}\PYG{o}{/}\PYG{n}{CPU}\PYG{o}{/}\PYG{n}{Configuration} \PYG{n}{Selection} \PYG{p}{(}\PYG{n}{Nordic} \PYG{n}{Semiconductor} \PYG{n}{nRF52} \PYG{n}{series} \PYG{n}{MCU}\PYG{p}{)}  \PYG{o}{\PYGZhy{}}\PYG{o}{\PYGZhy{}}\PYG{o}{\PYGZhy{}}\PYG{o}{\PYGZgt{}}
    \PYG{n}{Hardware} \PYG{n}{Configuration}  \PYG{o}{\PYGZhy{}}\PYG{o}{\PYGZhy{}}\PYG{o}{\PYGZhy{}}\PYG{o}{\PYGZgt{}}
    \PYG{n}{ARM} \PYG{n}{Options}  \PYG{o}{\PYGZhy{}}\PYG{o}{\PYGZhy{}}\PYG{o}{\PYGZhy{}}\PYG{o}{\PYGZgt{}}
    \PYG{n}{Architecture} \PYG{p}{(}\PYG{n}{ARM} \PYG{n}{architecture}\PYG{p}{)}  \PYG{o}{\PYGZhy{}}\PYG{o}{\PYGZhy{}}\PYG{o}{\PYGZhy{}}\PYG{o}{\PYGZgt{}}
    \PYG{n}{General} \PYG{n}{Architecture} \PYG{n}{Options}  \PYG{o}{\PYGZhy{}}\PYG{o}{\PYGZhy{}}\PYG{o}{\PYGZhy{}}\PYG{o}{\PYGZgt{}}
\PYG{p}{[} \PYG{p}{]} \PYG{n}{Floating} \PYG{n}{point}  \PYG{o}{\PYGZhy{}}\PYG{o}{\PYGZhy{}}\PYG{o}{\PYGZhy{}}\PYG{o}{\PYGZhy{}}
    \PYG{n}{General} \PYG{n}{Kernel} \PYG{n}{Options}  \PYG{o}{\PYGZhy{}}\PYG{o}{\PYGZhy{}}\PYG{o}{\PYGZhy{}}\PYG{o}{\PYGZgt{}}
    \PYG{n}{Device} \PYG{n}{Drivers}  \PYG{o}{\PYGZhy{}}\PYG{o}{\PYGZhy{}}\PYG{o}{\PYGZhy{}}\PYG{o}{\PYGZgt{}} \PYG{o}{*}\PYG{o}{*}\PYG{o}{*}\PYG{o}{*}\PYG{o}{*}\PYG{o}{*}\PYG{o}{*}\PYG{o}{*}\PYG{o}{*}\PYG{o}{*}\PYG{o}{*}\PYG{o}{*}\PYG{o}{*}\PYG{o}{*}\PYG{n}{SELECT} \PYG{n}{THIS} \PYG{n}{ONE}\PYG{o}{*}\PYG{o}{*}\PYG{o}{*}\PYG{o}{*}\PYG{o}{*}\PYG{o}{*}\PYG{o}{*}\PYG{o}{*}\PYG{o}{*}\PYG{o}{*}\PYG{o}{*}\PYG{o}{*}\PYG{o}{*}\PYG{o}{*}\PYG{o}{*}\PYG{o}{*}\PYG{o}{*}\PYG{o}{*}\PYG{o}{*}\PYG{o}{*}\PYG{o}{*}\PYG{o}{*}\PYG{o}{*}\PYG{o}{*}\PYG{o}{*}\PYG{o}{*}\PYG{o}{*}
    \PYG{n}{C} \PYG{n}{Library}  \PYG{o}{\PYGZhy{}}\PYG{o}{\PYGZhy{}}\PYG{o}{\PYGZhy{}}\PYG{o}{\PYGZgt{}}
    \PYG{n}{Additional} \PYG{n}{libraries}  \PYG{o}{\PYGZhy{}}\PYG{o}{\PYGZhy{}}\PYG{o}{\PYGZhy{}}\PYG{o}{\PYGZgt{}}
\PYG{p}{[}\PYG{o}{*}\PYG{p}{]} \PYG{n}{Bluetooth}  \PYG{o}{\PYGZhy{}}\PYG{o}{\PYGZhy{}}\PYG{o}{\PYGZhy{}}\PYG{o}{\PYGZgt{}}
\PYG{p}{[} \PYG{p}{]} \PYG{n}{Console} \PYG{n}{subsystem}\PYG{o}{/}\PYG{n}{support} \PYG{n}{routines} \PYG{p}{[}\PYG{n}{EXPERIMENTAL}\PYG{p}{]}  \PYG{o}{\PYGZhy{}}\PYG{o}{\PYGZhy{}}\PYG{o}{\PYGZhy{}}\PYG{o}{\PYGZhy{}}
\PYG{p}{[} \PYG{p}{]} \PYG{n}{C}\PYG{o}{+}\PYG{o}{+} \PYG{n}{support} \PYG{k}{for} \PYG{n}{the} \PYG{n}{application}  \PYG{o}{\PYGZhy{}}\PYG{o}{\PYGZhy{}}\PYG{o}{\PYGZhy{}}\PYG{o}{\PYGZhy{}}
    \PYG{n}{System} \PYG{n}{Monitoring} \PYG{n}{Options}  \PYG{o}{\PYGZhy{}}\PYG{o}{\PYGZhy{}}\PYG{o}{\PYGZhy{}}\PYG{o}{\PYGZgt{}}
    \PYG{n}{Debugging} \PYG{n}{Options}  \PYG{o}{\PYGZhy{}}\PYG{o}{\PYGZhy{}}\PYG{o}{\PYGZhy{}}\PYG{o}{\PYGZgt{}}
\PYG{p}{[} \PYG{p}{]} \PYG{n}{Disk} \PYG{n}{Interface}  \PYG{o}{\PYGZhy{}}\PYG{o}{\PYGZhy{}}\PYG{o}{\PYGZhy{}}\PYG{o}{\PYGZhy{}}
    \PYG{n}{File} \PYG{n}{Systems}  \PYG{o}{\PYGZhy{}}\PYG{o}{\PYGZhy{}}\PYG{o}{\PYGZhy{}}\PYG{o}{\PYGZgt{}}
\PYG{o}{\PYGZhy{}}\PYG{o}{*}\PYG{o}{\PYGZhy{}} \PYG{n}{Logging}  \PYG{o}{\PYGZhy{}}\PYG{o}{\PYGZhy{}}\PYG{o}{\PYGZhy{}}\PYG{o}{\PYGZgt{}}
    \PYG{n}{Management}  \PYG{o}{\PYGZhy{}}\PYG{o}{\PYGZhy{}}\PYG{o}{\PYGZhy{}}\PYG{o}{\PYGZgt{}}
    \PYG{n}{Networking}  \PYG{o}{\PYGZhy{}}\PYG{o}{\PYGZhy{}}\PYG{o}{\PYGZhy{}}\PYG{o}{\PYGZgt{}}
\end{sphinxVerbatim}

\begin{sphinxVerbatim}[commandchars=\\\{\}]
\PYG{p}{[} \PYG{p}{]} \PYG{n}{IEEE} \PYG{l+m+mf}{802.15}\PYG{o}{.}\PYG{l+m+mi}{4} \PYG{n}{drivers} \PYG{n}{options}  \PYG{o}{\PYGZhy{}}\PYG{o}{\PYGZhy{}}\PYG{o}{\PYGZhy{}}\PYG{o}{\PYGZhy{}}
\PYG{p}{(}\PYG{n}{UART\PYGZus{}0}\PYG{p}{)} \PYG{n}{Device} \PYG{n}{Name} \PYG{n}{of} \PYG{n}{UART} \PYG{n}{Device} \PYG{k}{for} \PYG{n}{UART} \PYG{n}{Console}
\PYG{p}{[}\PYG{o}{*}\PYG{p}{]} \PYG{n}{Console} \PYG{n}{drivers}  \PYG{o}{\PYGZhy{}}\PYG{o}{\PYGZhy{}}\PYG{o}{\PYGZhy{}}\PYG{o}{\PYGZgt{}}
\PYG{p}{[} \PYG{p}{]} \PYG{n}{Net} \PYG{n}{loopback} \PYG{n}{driver}  \PYG{o}{\PYGZhy{}}\PYG{o}{\PYGZhy{}}\PYG{o}{\PYGZhy{}}\PYG{o}{\PYGZhy{}}
\PYG{p}{[}\PYG{o}{*}\PYG{p}{]} \PYG{n}{Serial} \PYG{n}{Drivers}  \PYG{o}{\PYGZhy{}}\PYG{o}{\PYGZhy{}}\PYG{o}{\PYGZhy{}}\PYG{o}{\PYGZgt{}}
    \PYG{n}{Interrupt} \PYG{n}{Controllers}  \PYG{o}{\PYGZhy{}}\PYG{o}{\PYGZhy{}}\PYG{o}{\PYGZhy{}}\PYG{o}{\PYGZgt{}}
    \PYG{n}{Timer} \PYG{n}{Drivers}  \PYG{o}{\PYGZhy{}}\PYG{o}{\PYGZhy{}}\PYG{o}{\PYGZhy{}}\PYG{o}{\PYGZgt{}}
\PYG{o}{\PYGZhy{}}\PYG{o}{*}\PYG{o}{\PYGZhy{}} \PYG{n}{Entropy} \PYG{n}{Drivers}  \PYG{o}{\PYGZhy{}}\PYG{o}{\PYGZhy{}}\PYG{o}{\PYGZhy{}}\PYG{o}{\PYGZgt{}}
\PYG{p}{[}\PYG{o}{*}\PYG{p}{]} \PYG{n}{GPIO} \PYG{n}{Drivers}  \PYG{o}{\PYGZhy{}}\PYG{o}{\PYGZhy{}}\PYG{o}{\PYGZhy{}}\PYG{o}{\PYGZgt{}}
\PYG{p}{[} \PYG{p}{]} \PYG{n}{Shared} \PYG{n}{interrupt} \PYG{n}{driver}  \PYG{o}{\PYGZhy{}}\PYG{o}{\PYGZhy{}}\PYG{o}{\PYGZhy{}}\PYG{o}{\PYGZhy{}}
\PYG{p}{[} \PYG{p}{]} \PYG{n}{SPI} \PYG{n}{hardware} \PYG{n}{bus} \PYG{n}{support}  \PYG{o}{\PYGZhy{}}\PYG{o}{\PYGZhy{}}\PYG{o}{\PYGZhy{}}\PYG{o}{\PYGZhy{}}
\PYG{p}{[} \PYG{p}{]} \PYG{n}{I2C} \PYG{n}{Drivers}  \PYG{o}{\PYGZhy{}}\PYG{o}{\PYGZhy{}}\PYG{o}{\PYGZhy{}}\PYG{o}{\PYGZhy{}}
\PYG{p}{[} \PYG{p}{]} \PYG{n}{I2S} \PYG{n}{bus} \PYG{n}{drivers}  \PYG{o}{\PYGZhy{}}\PYG{o}{\PYGZhy{}}\PYG{o}{\PYGZhy{}}\PYG{o}{\PYGZhy{}}
\PYG{p}{[} \PYG{p}{]} \PYG{n}{PWM} \PYG{p}{(}\PYG{n}{Pulse} \PYG{n}{Width} \PYG{n}{Modulation}\PYG{p}{)} \PYG{n}{Drivers}  \PYG{o}{\PYGZhy{}}\PYG{o}{\PYGZhy{}}\PYG{o}{\PYGZhy{}}\PYG{o}{\PYGZhy{}}
\PYG{p}{[} \PYG{p}{]} \PYG{n}{Enable} \PYG{n}{board} \PYG{n}{pinmux} \PYG{n}{driver}  \PYG{o}{\PYGZhy{}}\PYG{o}{\PYGZhy{}}\PYG{o}{\PYGZhy{}}\PYG{o}{\PYGZhy{}}
\PYG{p}{[} \PYG{p}{]} \PYG{n}{ADC} \PYG{n}{drivers}  \PYG{o}{\PYGZhy{}}\PYG{o}{\PYGZhy{}}\PYG{o}{\PYGZhy{}}\PYG{o}{\PYGZhy{}}
\PYG{p}{[} \PYG{p}{]} \PYG{n}{Watchdog} \PYG{n}{Support}  \PYG{o}{\PYGZhy{}}\PYG{o}{\PYGZhy{}}\PYG{o}{\PYGZhy{}}\PYG{o}{\PYGZhy{}}
\PYG{p}{[}\PYG{o}{*}\PYG{p}{]} \PYG{n}{Hardware} \PYG{n}{clock} \PYG{n}{controller} \PYG{n}{support}  \PYG{o}{\PYGZhy{}}\PYG{o}{\PYGZhy{}}\PYG{o}{\PYGZhy{}}\PYG{o}{\PYGZgt{}} \PYG{o}{\PYGZlt{}\PYGZlt{}}\PYG{o}{\PYGZlt{}\PYGZlt{}}\PYG{o}{\PYGZlt{}\PYGZlt{}}\PYG{o}{\PYGZlt{}\PYGZlt{}}\PYG{o}{\PYGZlt{}\PYGZlt{}}\PYG{o}{\PYGZlt{}\PYGZlt{}}\PYG{o}{\PYGZlt{}\PYGZlt{}}\PYG{o}{\PYGZlt{}\PYGZlt{}}\PYG{n}{SELECT} \PYG{n}{THIS} \PYG{n}{ONE}\PYG{o}{\PYGZlt{}\PYGZlt{}}\PYG{o}{\PYGZlt{}\PYGZlt{}}\PYG{o}{\PYGZlt{}\PYGZlt{}}\PYG{o}{\PYGZlt{}\PYGZlt{}}\PYG{o}{\PYGZlt{}\PYGZlt{}}
\PYG{p}{[} \PYG{p}{]} \PYG{n}{Precision} \PYG{n}{Time} \PYG{n}{Protocol} \PYG{n}{Clock} \PYG{n}{driver} \PYG{n}{support}
\PYG{p}{[} \PYG{p}{]} \PYG{n}{IPM} \PYG{n}{drivers}  \PYG{o}{\PYGZhy{}}\PYG{o}{\PYGZhy{}}\PYG{o}{\PYGZhy{}}\PYG{o}{\PYGZhy{}}
    \PYG{n}{Max} \PYG{n}{compiled}\PYG{o}{\PYGZhy{}}\PYG{o+ow}{in} \PYG{n}{log} \PYG{n}{level} \PYG{k}{for} \PYG{n}{ipm} \PYG{p}{(}\PYG{n}{Info}\PYG{p}{)}  \PYG{o}{\PYGZhy{}}\PYG{o}{\PYGZhy{}}\PYG{o}{\PYGZhy{}}\PYG{o}{\PYGZgt{}}
\PYG{p}{[} \PYG{p}{]} \PYG{n}{Flash} \PYG{n}{hardware} \PYG{n}{support}  \PYG{o}{\PYGZhy{}}\PYG{o}{\PYGZhy{}}\PYG{o}{\PYGZhy{}}\PYG{o}{\PYGZhy{}}
\PYG{p}{[} \PYG{p}{]} \PYG{n}{Sensor} \PYG{n}{Drivers}  \PYG{o}{\PYGZhy{}}\PYG{o}{\PYGZhy{}}\PYG{o}{\PYGZhy{}}\PYG{o}{\PYGZhy{}}
\end{sphinxVerbatim}

\begin{sphinxVerbatim}[commandchars=\\\{\}]
    \PYG{n}{Max} \PYG{n}{compiled}\PYG{o}{\PYGZhy{}}\PYG{o+ow}{in} \PYG{n}{log} \PYG{n}{level} \PYG{k}{for} \PYG{n}{clock} \PYG{n}{control} \PYG{p}{(}\PYG{n}{Info}\PYG{p}{)}  \PYG{o}{\PYGZhy{}}\PYG{o}{\PYGZhy{}}\PYG{o}{\PYGZhy{}}\PYG{o}{\PYGZgt{}}
\PYG{p}{[}\PYG{o}{*}\PYG{p}{]} \PYG{n}{NRF} \PYG{n}{Clock} \PYG{n}{controller} \PYG{n}{support}  \PYG{o}{\PYGZhy{}}\PYG{o}{\PYGZhy{}}\PYG{o}{\PYGZhy{}}\PYG{o}{\PYGZgt{}} \PYG{o}{\PYGZlt{}\PYGZlt{}}\PYG{o}{\PYGZlt{}\PYGZlt{}}\PYG{o}{\PYGZlt{}\PYGZlt{}}\PYG{o}{\PYGZlt{}\PYGZlt{}}\PYG{o}{\PYGZlt{}\PYGZlt{}}\PYG{o}{\PYGZlt{}\PYGZlt{}}\PYG{o}{\PYGZlt{}\PYGZlt{}}\PYG{o}{\PYGZlt{}\PYGZlt{}}\PYG{o}{\PYGZlt{}\PYGZlt{}}\PYG{o}{\PYGZlt{}\PYGZlt{}}\PYG{n}{SELECT} \PYG{n}{THIS} \PYG{n}{ONE}\PYG{o}{\PYGZlt{}\PYGZlt{}}\PYG{o}{\PYGZlt{}\PYGZlt{}}\PYG{o}{\PYGZlt{}\PYGZlt{}}\PYG{o}{\PYGZlt{}\PYGZlt{}}\PYG{o}{\PYGZlt{}\PYGZlt{}}\PYG{o}{\PYGZlt{}}
\end{sphinxVerbatim}


\chapter{hacking   the pinetime smartwatch}
\label{\detokenize{flashing:hacking-the-pinetime-smartwatch}}\label{\detokenize{flashing::doc}}
\begin{sphinxVerbatim}[commandchars=\\\{\}]
\PYG{n}{The} \PYG{n}{pinetime} \PYG{o+ow}{is} \PYG{n}{preloaded} \PYG{k}{with} \PYG{n}{firmware}\PYG{o}{.}
\PYG{n}{This} \PYG{n}{firmware} \PYG{o+ow}{is} \PYG{n}{secured}\PYG{p}{,} \PYG{n}{you} \PYG{n}{cannot} \PYG{n}{peek} \PYG{n}{into} \PYG{n}{it}\PYG{o}{.}
\end{sphinxVerbatim}

\begin{sphinxadmonition}{note}{Note:}
The pinetime has a swd interface.
To be able to write firmware,  you need special hardware.
I use a stm-link which is very cheap(2\$).
You can also use the GPIO header of a raspberry pi.
(my repo: \sphinxurl{https://github.com/najnesnaj/openocd} is adapted for the orange pi)
\end{sphinxadmonition}

To flash the software I use openocd :
example for stm-link usb-stick

\begin{sphinxVerbatim}[commandchars=\\\{\}]
\PYG{g+gp}{\PYGZsh{}} openocd \PYGZhy{}s /usr/local/share/openocd/scripts \PYGZhy{}f interface/stlink.cfg \PYGZhy{}f target/nrf52.cfg
\end{sphinxVerbatim}

example for the orange-pi GPIO header (or raspberry)
\begin{quote}

\# openocd -f /usr/local/share/openocd/scripts/interface/sysfsgpio-raspberrypi.cfg
-c ‘transport select swd’ -f /usr/local/share/openocd/scripts/target/nrf52.cfg
-c ‘bindto 0.0.0.0’
\end{quote}

once you started the openocd background server, you can connect to it using:

\begin{sphinxVerbatim}[commandchars=\\\{\}]
\PYG{g+gp}{\PYGZsh{}}telnet \PYG{l+m}{127}.0.0.1 \PYG{l+m}{4444}
\end{sphinxVerbatim}

programming

\begin{sphinxVerbatim}[commandchars=\\\{\}]
\PYG{g+go}{once your telnet sessions started:}
\PYG{g+go}{Trying 127.0.0.1...}
\PYG{g+go}{Connected to 127.0.0.1.}
\PYG{g+go}{Escape character is \PYGZsq{}\PYGZca{}]\PYGZsq{}.}
\PYG{g+go}{Open On\PYGZhy{}Chip Debugger}
\PYG{g+gp}{\PYGZgt{}} program zephyr.bin

\PYG{g+go}{target halted due to debug\PYGZhy{}request, current mode: Thread}
\PYG{g+go}{xPSR: 0x01000000 pc: 0x00001534 msp: 0x20004a10}
\PYG{g+go}{** Programming Started **}
\PYG{g+go}{auto erase enabled}
\PYG{g+go}{using fast async flash loader. This is currently supported}
\PYG{g+go}{only with ST\PYGZhy{}Link and CMSIS\PYGZhy{}DAP. If you have issues, add}
\PYG{g+go}{\PYGZdq{}set WORKAREASIZE 0\PYGZdq{} before sourcing nrf51.cfg/nrf52.cfg to disable it}
\PYG{g+go}{target halted due to breakpoint, current mode: Thread}
\PYG{g+go}{xPSR: 0x61000000 pc: 0x2000001e msp: 0x20004a10}
\PYG{g+go}{wrote 24576 bytes from file zephyr.bin in 1.703540s (14.088 KiB/s)}
\PYG{g+go}{** Programming Finished **}

\PYG{g+go}{And finally execute a reset :}
\PYG{g+gp}{\PYGZgt{}}reset
\end{sphinxVerbatim}

removing write protection see:   {\hyperref[\detokenize{writeprotection:flashing}]{\sphinxcrossref{\DUrole{std,std-ref}{howto flash your zephyr image}}}}


\chapter{debugging   the pinetime smartwatch}
\label{\detokenize{debugging:debugging-the-pinetime-smartwatch}}\label{\detokenize{debugging::doc}}
\begin{sphinxVerbatim}[commandchars=\\\{\}]
\PYG{n}{The} \PYG{n}{pinetime} \PYG{n}{does} \PYG{o+ow}{not} \PYG{n}{have} \PYG{n}{a} \PYG{n}{serial} \PYG{n}{port}\PYG{o}{.}
\PYG{n}{I} \PYG{n}{do} \PYG{o+ow}{not} \PYG{n}{have} \PYG{n}{a} \PYG{n}{segger} \PYG{n}{debugging} \PYG{n}{probe}\PYG{o}{.}
\PYG{n}{A} \PYG{n}{way} \PYG{n}{around} \PYG{n}{this}\PYG{p}{,} \PYG{n}{it} \PYG{n}{to} \PYG{n}{put} \PYG{n}{a} \PYG{n}{value} \PYG{o+ow}{in} \PYG{n}{memory} \PYG{n}{at} \PYG{n}{a} \PYG{n}{fixed} \PYG{n}{location}\PYG{o}{.}
\PYG{n}{With} \PYG{n}{openocd} \PYG{n}{you} \PYG{n}{can} \PYG{n}{peek} \PYG{n}{at} \PYG{n}{this} \PYG{n}{memory} \PYG{n}{location}\PYG{o}{.}
\end{sphinxVerbatim}

\begin{sphinxadmonition}{note}{Note:}
\#define MY\_REGISTER (*(volatile uint8\_t*)0x2000F000)

in the program you can set values:
MY\_REGISTER=1;
MY\_REGISTER=8;

this way you know till where the code executes
\end{sphinxadmonition}

\begin{sphinxVerbatim}[commandchars=\\\{\}]
\PYG{g+gp}{\PYGZsh{}}telnet \PYG{l+m}{127}.0.0.1 \PYG{l+m}{4444}
\end{sphinxVerbatim}

programming

\begin{sphinxVerbatim}[commandchars=\\\{\}]
\PYG{g+go}{once your telnet sessions started:}
\PYG{g+go}{Trying 127.0.0.1...}
\PYG{g+go}{Connected to 127.0.0.1.}
\PYG{g+go}{Escape character is \PYGZsq{}\PYGZca{}]\PYGZsq{}.}
\PYG{g+go}{Open On\PYGZhy{}Chip Debugger}
\PYG{g+gp}{\PYGZgt{}}mdw 0x2000F000 0x1

\PYG{g+go}{the last byte shows the value of your program trace value}
\end{sphinxVerbatim}


\chapter{scanning the I2C\_1 port}
\label{\detokenize{i2cscanning:scanning-the-i2c-1-port}}\label{\detokenize{i2cscanning::doc}}
\begin{sphinxVerbatim}[commandchars=\\\{\}]
\PYG{n}{The} \PYG{n}{pinetime} \PYG{n}{does} \PYG{o+ow}{not} \PYG{n}{have} \PYG{n}{a} \PYG{n}{serial} \PYG{n}{port}\PYG{o}{.}
\PYG{n}{I} \PYG{n}{do} \PYG{o+ow}{not} \PYG{n}{have} \PYG{n}{a} \PYG{n}{segger} \PYG{n}{debugging} \PYG{n}{probe}\PYG{o}{.}
\PYG{n}{A} \PYG{n}{way} \PYG{n}{around} \PYG{n}{this}\PYG{p}{,} \PYG{n}{it} \PYG{n}{to} \PYG{n}{put} \PYG{n}{a} \PYG{n}{value} \PYG{o+ow}{in} \PYG{n}{memory} \PYG{n}{at} \PYG{n}{a} \PYG{n}{fixed} \PYG{n}{location}\PYG{o}{.}
\PYG{n}{With} \PYG{n}{openocd} \PYG{n}{you} \PYG{n}{can} \PYG{n}{peek} \PYG{n}{at} \PYG{n}{this} \PYG{n}{memory} \PYG{n}{location}\PYG{o}{.}
\end{sphinxVerbatim}


\section{Building and Running}
\label{\detokenize{i2cscanning:building-and-running}}
In this repo under samples you will find an adapted i2c scanner program.

\begin{sphinxVerbatim}[commandchars=\\\{\}]
\PYG{g+go}{west build \PYGZhy{}p \PYGZhy{}b pinetime samples/drivers/i2c\PYGZus{}scanner}
\end{sphinxVerbatim}

\begin{sphinxadmonition}{note}{Note:}
\#define MY\_REGISTER (*(volatile uint8\_t*)0x2000F000)

in the program you can set values:
MY\_REGISTER=1;
MY\_REGISTER=8;

this way you know till where the code executes
\end{sphinxadmonition}

\begin{sphinxVerbatim}[commandchars=\\\{\}]
\PYG{g+gp}{\PYGZsh{}}telnet \PYG{l+m}{127}.0.0.1 \PYG{l+m}{4444}
\end{sphinxVerbatim}

Peeking

\begin{sphinxVerbatim}[commandchars=\\\{\}]
\PYG{g+go}{once your telnet sessions started:}
\PYG{g+go}{Trying 127.0.0.1...}
\PYG{g+go}{Connected to 127.0.0.1.}
\PYG{g+go}{Escape character is \PYGZsq{}\PYGZca{}]\PYGZsq{}.}
\PYG{g+go}{Open On\PYGZhy{}Chip Debugger}
\PYG{g+gp}{\PYGZgt{}}mdw 0x2000F000 0x1
\PYG{g+go}{0x2000f000: 00c24418}
\end{sphinxVerbatim}

\sphinxstyleemphasis{Note:}:

this corresponds to 0x18, 0x44 and 0xC2 (which is endvalue of scanner, so it does not detect touchscreen, which should be touched first….)


\chapter{howto flash your zephyr image}
\label{\detokenize{writeprotection:howto-flash-your-zephyr-image}}\label{\detokenize{writeprotection:flashing}}\label{\detokenize{writeprotection::doc}}
Once you completed your \sphinxcode{\sphinxupquote{west build}} , your image is located under the build directory

\begin{sphinxVerbatim}[commandchars=\\\{\}]
\PYG{g+gp}{\PYGZdl{}} \PYG{n+nb}{cd} \PYGZti{}/zephyrproject/zephyr/build/zephyr
\PYG{g+go}{here you can find zephyr.bin which you can flash}
\end{sphinxVerbatim}

I have an orange pi (single board computer) in my network.

I copy the image using \$scp -P 8888 zephyr.bin 192.168.0.77:/usr/src/pinetime
(secure copy using my user defined port 8888 which is normally port 22)

\begin{sphinxadmonition}{note}{Note:}
the PineTime watch is read/write protected
executing the following : nrf52.dap apreg 1 0x0c shows 0x0

Mind you st-link does not allow you to execute that command, you need J-link.
There is a workaround using the GPIO of a raspberry pi or a Orangepi.
You have to reconfigure Openocd with the \textendash{}enable-cmsis-dap option.

Unlock the chip by executing the command:
\textgreater{} nrf52.dap apreg 1 0x04 0x01
\end{sphinxadmonition}


\chapter{howto generate pdf documents}
\label{\detokenize{latexpdf:howto-generate-pdf-documents}}\label{\detokenize{latexpdf::doc}}
sphinx cannot generate pdf directly, and needs latex

\begin{sphinxVerbatim}[commandchars=\\\{\}]
\PYG{n}{apt}\PYG{o}{\PYGZhy{}}\PYG{n}{get} \PYG{n}{install} \PYG{n}{latexmk}
\PYG{n}{apt}\PYG{o}{\PYGZhy{}}\PYG{n}{get} \PYG{n}{install} \PYG{n}{texlive}\PYG{o}{\PYGZhy{}}\PYG{n}{fonts}\PYG{o}{\PYGZhy{}}\PYG{n}{recommended}
\PYG{n}{apt}\PYG{o}{\PYGZhy{}}\PYG{n}{get} \PYG{n}{install} \PYG{n}{xzdec}
\PYG{n}{apt}\PYG{o}{\PYGZhy{}}\PYG{n}{get} \PYG{n}{install} \PYG{n}{cmap}
\PYG{n}{apt}\PYG{o}{\PYGZhy{}}\PYG{n}{get} \PYG{n}{install} \PYG{n}{texlive}\PYG{o}{\PYGZhy{}}\PYG{n}{latex}\PYG{o}{\PYGZhy{}}\PYG{n}{recommended}
\PYG{n}{apt}\PYG{o}{\PYGZhy{}}\PYG{n}{get} \PYG{n}{install} \PYG{n}{texlive}\PYG{o}{\PYGZhy{}}\PYG{n}{latex}\PYG{o}{\PYGZhy{}}\PYG{n}{extra}
\end{sphinxVerbatim}



\renewcommand{\indexname}{Index}
\printindex
\end{document}